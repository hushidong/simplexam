
\begin{question}
用$4,2,0,7$这四个数字写出一个不读零的四位数是\answerblank{4270},写出一个读零
  的数字是\answerblank{4027}。
\end{question}


\begin{question}
小明有两件颜色不同的上衣和两条颜色不同的裤子,他可以有\answerblank{4}种不同
    的穿法。
\end{question}

\begin{question}
有语文、数学、品德三种书,小明、小丽、小红各拿一本;小明说:“我拿的是语文
  书”。小丽说:“我拿的不是数学书”。小红拿的是\answerblank{数学}书。
\end{question}

\begin{question}
给下面句子添加标点符号:
  “咦,是谁救了小白兔\answerblank{?}”小动物们说\answerblank{,}“真得谢谢他呢 \answerblank{!}”
\end{question}


\begin{question}
 若函数~$f(x)=x^{6m^2-5m-4}\,(m\in\mathbb{Z})$~的图像关于~$y$~轴对称,
  且~$f(2)<f(6)$, 则~$f(x)$~的解析式为\answerblank{$f(x)=x^{-4}$}.
\end{question}

\begin{question}
若~$f(x+1)=x^2\,(x\leq0)$, 则$f^{-1}(1)=$ \answerblank{$0$}.
\end{question}

\begin{question} ~$f(x)=1-\textbf{c}_8^1x+\textbf{c}_8^2x^2-\textbf{c}_8^3x^3+\cdots+\textbf{c}_8^8x^8$,
  则~$f\left(\tfrac{1}{2}+\tfrac{\sqrt{3}}{2}\textbf{i}\right)$ 的
  值是\answerblank{$-\tfrac{1}{2}-\tfrac{\sqrt{3}}{2}\textbf{i}$}.
\end{question}

\begin{question}
真空中无线电磁波传播速度为\answerblank{光速或 $3*10^8$ m/s}。	
\end{question}

\begin{question}
电磁波是发射到空中的能量,这种能量部分以电场的形式存在,部分的以\answerblank{磁场}的形式存在。	
\end{question}

\begin{question}
电磁波的基本特性有:速度、 方向、 极化、 强度、波长、\answerblank{频率}和相位。	
\end{question}

\begin{question}
线极化包括垂直和\answerblank{水平}极化。	
\end{question}


\begin{question}
已知二阶行列式 $\text{$\left|\begin{array}{cc}
  1 & 2\\
  - 3 & x
\end{array}\right|$=0}$,则 $x=$ \answerblank{$-6$}。
\end{question}

\begin{question}
五阶行列式的一共有 \answerblank{$120$} 项。
\end{question}

\begin{question}
向量组 $\alpha_1=(1,1,0), \alpha_2=(0,1,1), \alpha_3=(1,0,1)$,
则将向量 $\beta=(4, 5, 3)$ 表示为 $\alpha_1, \alpha_2, \alpha_3$
的线性组合为 $\beta=$ \answerblank{$3\alpha_1+2\alpha_2+\alpha_3$}。
\end{question}



\begin{question}
已知$P(A)=0.3$, $P(B|A)=0.4$, $P(B|\bar{A})=0.5$, 则$P(B)=$ \answerblank{$0.47$}。
\end{question}



\begin{question}
已知连续型$\xi$的密度函数为$\varphi(x)=\left\{
\begin{array}{ll}
  k \cos x, & - \frac{\pi}{2} < x < \frac{\pi}{2}\\
  0, & \text{其它}
\end{array}\right.$,
则$k=$ \answerblank{$\dfrac{1}{2}$}。
\end{question}


\begin{question}
已知随机变量$\xi$的期望和方差各为$E\xi=3, D\xi=2$, 则$E\xi^2=$ \answerblank{$11$}。
\end{question}



\begin{question}
电子管寿命$\xi$满足平均寿命为$1000$小时的指数分布,则它的寿命小于$2000$小时概率为 \answerblank{$1-e^{-2}$}。
\end{question}



\begin{question}
已知$\xi$和$\eta$相互独立且$\xi\sim N(1,4), \eta\sim N(2,5)$,则$\xi-2\eta\sim$ \answerblank{$N(-3,24)$}。
\end{question} 