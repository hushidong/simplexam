\documentclass[answer=true,countunit=section,sheetsize=A3,paperprint=double,scoretable]{simplexam}

\begin{document}

\begin{center}
\zihao{-3}\heiti  2018年火星高等学校招生统一考试\\ %
\zihao{4}    文理综合 %
\end{center}

%{\bfseries \heiti 注意事项}:
%\begin{enumerate}
%\item 作弊请不要被老师发现。
%\item 一旦被发现作弊,不要试图以空口求情或小恩小惠来逃避处罚。
%\item “每个人都有一个他无法拒绝的价格。”
%\end{enumerate}

\iftoggle{auxinfoused}{\assesstaball}{}

\section{判断题:本大题共4小题,每小题5分,共20分。判断下列各题,正确的在题后括号内打\ding{52},错的打\ding{56}。}

\includequestions{questionliba}%[random=3]

%测试answer的换行
\begin{question}
尼采说道:“世上只有一个基督徒,他已经死了”,这句话中所说的“他”是指彼得吗?
你好好好 \answer{wrong}
\end{question}

\section{选择题:本大题共4小题,每小题10分,共40分。在每小题给出的四个选项中,只有一项是符合题目要求的。}

\includequestions[random=3]{questionlibb}%[random=3]


\section{填空题:本题共4小题,每小题5分,共20分。}

%测试blank命令中中文字符和数学字符连在一起时可能出现错误
\begin{question}
真空中无线电磁波传播速度为\answerblank{光速或 $3*10^8$m/s}。	
\end{question}


\includequestions[random=3]{questionlibc}%[random=3]


\section{简答题。}


\includequestions[random=3]{questionlibd}%


\section{填字题:本题共4小题,每小题5分,共20分。}

%如果要增加汉字之间的间距,可以设置CJKglue,但必须要在字号命令改变命令之后设置
\begin{question}
请根据拼音把句中的汉字填写出来。\vskip 1em

\Large
\xeCJKsetup{
    CJKglue = {\hskip 0.2em plus .1em}
}

段落中的\answerhanzi{拼音}文字	\answerpoints
\end{question}

%对比上一题可以看到,先设glue后设字号,无法增加汉字间距
\begin{question}
请根据米字格中汉字把拼音填写出来。\vskip 1em

\xeCJKsetup{
    CJKglue = {\hskip 0.2em plus .1em}
}

\Large

段落中的\answerpinyin{拼音}文字	\answerpoints
\end{question}


\begin{question}
请根据拼音把下面的唐诗填写出来。\vskip 1em

\centering
\huge

\answerhanzi{春晓}

\Large

\answerhanzi{孟浩然}

\answerhanzi{春眠不觉晓},\answerhanzi{处处闻啼鸟}。

\answerhanzi{夜来风雨声},\answerhanzi{花落知多少}。

\answerpoints
\end{question}


\section{作文题。阅读下面黑格尔《历史哲学》中的一段话,思考并提出你对这段话的理解话的理解}

\SetupExSheets{
    headings ={runin-nonumber},
}

\begin{question}
\begin{quotation}
  \large \kaishu 经验或曰历史给我们的教训却是,人民和政府从来就没有从历史学到任何东西,从未依照本应从从历史中抽绎出来的教训行事。每个时代都有它特殊的处境,都具有一种个别的情况,使它的举动行事,不得不全由自己来考虑、自己来解决。当重大事件纷乘交迫的时候,一般笼统的信条毫无裨益。回忆过去的同样情形,也是徒劳无功的。一个灰色的回忆不能抗衡``现在''的生动和自由。
\end{quotation}
\answerpoints
\end{question}

\drawcomposition[0.7]{20}{13}
%
\end{document}
