\documentclass[answer=true,countunit=section,sheetsize=A3,paperprint=double,scoretable]{simplexam}

%页眉信息
\pageheading{试\hspace{4em}卷\hspace{4em}标\hspace{4em}题}

%页脚信息
\pagefooting{第\thepage 页 共\totalpage 页; \today}

%学生信息-放密封区
\pagestudent{学院:\underline{\hspace{5em}}班级:\underline{\hspace{5em}}
  姓名:\underline{\hspace{5em}}学号:\underline{\hspace{5em}}}

%密封线信息
\pagesealing{\small 学\\[1em]生\\[1em]答\\[1em]卷\\[1em]不\\[1em]要\\[1em]超\\[1em]过\\[1em]此\\[1em]线}


\begin{document}

%试卷信息直接写
\begin{center}
\zihao{2}\heiti  2019年地球高等学校招生统一考试\\ %
\zihao{-3}    文理综合 %
\end{center}

%卷面上的应试者信息,直接写
\begin{center}
\zihao{4}\fangsong 单位\underline{\hspace{6em}}姓名\underline{\hspace{6em}}
得分\underline{\hspace{6em}} %
\end{center}

%注意事项,直接写
%{\bfseries \heiti 注意事项}:
%\begin{enumerate}
%\item 作弊请不要被老师发现。
%\item 一旦被发现作弊,不要试图以空口求情或小恩小惠来逃避处罚。
%\item “每个人都有一个他无法拒绝的价格。”
%\end{enumerate}

%
%给出总的得分表
%
\assesstaball

\section{判断题:共5小题,每小题2分,共10分。判断下列各题,
正确的在题后括号内打钩,错的打叉。}

{

\Setquestiondefaultpts{2}

\includequestions[random=4]{questionliba}%[random=3]

%测试answer的换行
\begin{question}
尼采说道:“世上只有一个基督徒,他已经死了”,这句话中所说的“他”是指彼得吗?
你好好好 \answer[0]{wrong}
\end{question}

}

\section{选择题:共5小题,每小题3分,共15分。在每小题给出的四个选项中,只有一项是符合题目要求的。}

{

\Setquestiondefaultpts{3}
\includequestions[random=5]{questionlibb}%[random=3]

}

\section{填空题:共5小题,每小题3分,共15分。}

{

\Setquestiondefaultpts{3}

%测试blank命令中中文字符和数学字符连在一起时可能出现错误
\begin{question}
真空中无线电磁波传播速度为\answerblank{光速或 $3*10^8$m/s}。	
\end{question}


\includequestions[random=4]{questionlibc}%[random=3]

}

\section{简答题:共4小题,每小题5分,共20分。}

{

\Setquestiondefaultpts{5}

\includequestions[random=1]{questionlibd}%


\begin{question}
设数列$\{x_n\}$满足$x_1=\sqrt2$,$x_{n+1}=\sqrt{2+x_n}$.证明数列收敛,并求出极限。
\end{question}

\begin{solution}
事实上,由于$x_1<2$,且$x_k<2$时
\[x_{k+1}=\sqrt{2+x_k}<\sqrt{2+2}=2,\]
由数学归纳法知对所有$n$都有$x_n<2$,即数列有上界.
又由于
\[\frac{x_{n+1}}{x_n}=\sqrt{\frac{2}{x_n^2}+\frac{1}{x_n}}>\sqrt{\frac{2}{2^2}+\frac{1}{2}}=1,\]
所以数列单调增加.由极限存在准则II,数列必定收敛.\score{3}
设数列的极限为 $A$ ,对递推公式两边同时取极限得到
\[A=\sqrt{A+2}\]
解得$A=2$,即数列$\{x_n\}$的极限为$2$.\score{2}
\end{solution}


\begin{question}
设事件$A$和$B$相互独立,证明$A$和$\bar{B}$相互独立.
\end{question}

\begin{solution}
$P (A \cdot \bar{B}) = P (A - B) = P (A - A B)$ \score{1}
\qquad $= P (A) - P (A B) = P (A) - P (A) P (B)$ \score{1}
\qquad $= P (A) (1 - P (B)) = P (A) P (\bar{B})$ \score{0.5}
所以$A$和$\bar{B}$相互独立.\score{2}

\answerpoints[4]%{4.5}
\end{solution}


\begin{question}
用配方法将二次型 $f = x_1^2 + 2 x_1 x_2 - 6 x_1 x_3 + 2 x_2^2 - 12
x_2 x_3 + 9 x^2_3$ 化为标准形 $f = d_1 y^2_1 + d_2 y^2_2 + d_3 y^2_3$ 。
\end{question}

\begin{solution}{4.5}
$f = x_1^2 + 2 x_1 x_2 - 6 x_1 x_3 + 2 x_2^2 - 12 x_2 x_3 + 9 x^2_3$ \par
\qquad\qquad$= x_1^2 + 2 x_1 (x_2 - 3 x_3) + (x_2 - 3 x_3)^2 + x_2^2 - 6 x_2 x_3 $ \par
\qquad\qquad$= (x_1 + x_2 - 3 x_3)^2 + x_2^2 - 6 x_2 x_3$ \dotfill 1分 \par
\qquad\qquad$= (x_1 + x_2 - 3 x_3)^2 + x_2^2 - 2 x_2 \cdot 3 x_3 + (3 x_3)^2 - 9x_3^2$ \par
\qquad\qquad$= (x_1 + x_2 - 3 x_3)^2 + (x_2 - 3 x_3)^2 - 9 x_3^2$ \dotfill 1.5分\par
令$y_1 = x_1 + x_2 - 3 x_3, y_2 = x_2 - 3 x_3, y_3 = x_3$, \newline
则$f = y_1^2 + y_2^2 - 9y_3^2$为标准形。 \dotfill 2分
\end{solution}


}

\section{填字题:共4小题,每小题2.5分,共10分。}

{

\Setquestiondefaultpts{2.5}

%如果要增加汉字之间的间距,可以设置CJKglue,但必须要在字号命令改变命令之后设置
\begin{question}
请根据拼音把句中的汉字填写出来。\vskip 1em

\centering
\Large
\xeCJKsetup{
    CJKglue = {\hskip 0.2em plus .1em}
}

段落中的\answerhanzi{拼音}文字	\answerpoints[2]

\end{question}

%对比上一题可以看到,先设glue后设字号,无法增加汉字间距
\begin{question}
请根据米字格中汉字把拼音填写出来。\vskip 1em
\centering
\xeCJKsetup{
    CJKglue = {\hskip 0.2em plus .1em}
}

\Large

\answerpinyinbox{拼音文字}	\answerpoints[2.5]

\end{question}


\begin{question}
请根据汉字把拼音填写出来。\vskip 1em
\centering
\Large

\answerpinyin{白日依山\xpinyin{尽}{jin4}}	\answerpoints[2.5]

\end{question}


\begin{question}
请根据拼音把下面的唐诗填写出来。\vskip 1em

\centering
\huge

\answerhanzi{春晓}

\Large

\answerhanzi{孟浩然}

\answerhanzi{春眠不觉晓},\answerhanzi{处处闻啼鸟}。

\answerhanzi{夜来风雨声},\answerhanzi{花落知多少}。

\answerpoints[2.5]

\end{question}

}

\section{作文题:共30分。阅读下面黑格尔《历史哲学》中的一段话,思考并提出你对这段话的理解话的理解}

{

\Setquestiondefaultpts{30}

\SetupExSheets{
    headings ={runin-nonumber},
}

\begin{question}
\begin{quotation}
  \large \kaishu 经验或曰历史给我们的教训却是,人民和政府从来就没有从历史学到任何东西,从未依照本应从从历史中抽绎出来的教训行事。每个时代都有它特殊的处境,都具有一种个别的情况,使它的举动行事,不得不全由自己来考虑、自己来解决。当重大事件纷乘交迫的时候,一般笼统的信条毫无裨益。回忆过去的同样情形,也是徒劳无功的。一个灰色的回忆不能抗衡``现在''的生动和自由。
\end{quotation}

\answerpoints[26]

\end{question}



\drawcomposition[0.7]{20}{13}

}




\clearpage
{
\hyphenpenalty=100 %断词阈值, 值越大越不容易出现断词
\tolerance=10000 %丑度, 10000为最大无溢出盒子, 参考the texbook 第6章
\hbadness=100 %如果丑度超过hbadness这一阀值, 那么就会发出警告

\answersinfounit{1}

\answersinfounit{2}

\answersinfounit[5cm]{3}

\answersinfounit[14cm]{4}

\answersinfounit[14cm]{5}

}

%
\end{document}
