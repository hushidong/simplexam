

\begin{question}
土(产)——给括号中的字选择正确的解释 \answer{A}
  \begin{tasks}(2)
    \task 出自人工或天然的物品 \task 泛指房地或财物 \task 自然生长、人工制造或种
    植、出品
  \end{tasks}
\end{question}

\begin{question}
(如)此——给括号中的字选择正确的解释 \answer{A}
  \begin{tasks}(4)
    \task 似;好像 \task 及;比得上 \task 依照 \task 如果
  \end{tasks}
\end{question}

\begin{question}
已知双曲线 $C\colon\,\tfrac{x^2}{a^2}-\tfrac{y^2}{b^2}=1\,(a>0,b>0)$
  的一条渐近线方程为 $y=\tfrac{\sqrt{5}}{2}x$ ,且与椭
  圆$\tfrac{x^2}{12}+\tfrac{y^2}{3}=1$ 有公共焦点,则 $C$ 的方程为\answer{B}

  \begin{tasks}(4)
    \task $\tfrac{x^2}{8}-\tfrac{y^2}{10}=1$ \task $\tfrac{x^2}{4}-\tfrac{y^2}{5}=1$ \task $\tfrac{x^2}{5}-\tfrac{y^2}{4}=1$ \task $\tfrac{x^2}{4}-\tfrac{y^2}{3}=1$
  \end{tasks}
\end{question}

\begin{question}
某城市为了解游客人数的变化规律,提高旅游服务质量,收集并整理了2014年1
  月至2016年12月期间月接待游客量(单位:万人)的数据(图1).

  根据图\ref{fig:egimage:a},下列结论错误的是\answer{A}

  \begin{insertfig}{月接待游客量折线图}{fig:egimage:a}
    \includegraphics[width=0.4\linewidth]{example-image.pdf}
  \end{insertfig}%

  \begin{tasks}(1)
    \task 月接待游客量逐月增加
    \task 年接待游客量逐年增加
    \task 各年的月接待游客量高峰期大致在7,8月份
    \task 各年1月至6月的月接待游客量相对7月至12月,波动性更小,变化比较平稳
  \end{tasks}
\end{question}

\begin{question}
若~$a$、$b$~是直线, $\alpha$、$\beta$~是平面,
  则以下命题中真命题是\answer{D}

  \begin{tasks}(1)
    \task 若~$a$、$b$~异面, $a\subset\alpha$,$b\subset\beta$, 且~$a\perp b$, 则~$\alpha\perp\beta$
    \task 若~$a\parallel b$, $a\subset\alpha$, $b\subset\beta$,则~$\alpha\parallel\beta$
    \task 若~$a\parallel \alpha$, $b\subset\beta$, 则~$a$、$b$ 异面
    \task 若~$a\perp b$, $a\perp\alpha$,$b\perp\beta$, 则~$\alpha\perp\beta$
  \end{tasks}
\end{question}


\begin{question}
已知集合~$A=\{x\mid {x-1}<3 \}$,集合~$B=\{y|
  y=x^2+2x+1,x\in\mathbb{R}\}$, 则~$A\cap \complement_U B$~为\answer{C}

  \begin{tasks}(2)
    \task $[\,0,4)$
    \task $(-\infty,-2\,]\cup[4,+\infty)$
    \task $(-2,0)$
    \task $(0,4)$
  \end{tasks}
\end{question}


\begin{question}
下列各排列哪个是偶排列 \answer{D}
\options{3712456}
	{36715284}
	{654321}
	{41253}
\end{question}



\begin{question}
若三阶行列式 $\left|\begin{array}{ccc}
  a_1 & a_2 & a_3\\
  2 b_1 - a_1 & 2 b_2 - a_2 & 2 b_3 - a_3\\
  c_1 & c_2 & c_3
\end{array}\right| = 2$,则 $\left|\begin{array}{ccc}
  a_1 & a_2 & a_3\\
  b_1 & b_2 & b_3\\
  c_1 & c_2 & c_3
\end{array}\right|=$ \answer{A}
\options{1}
	{-1}
	{2}
	{-2}
\end{question}



\begin{question}
已知矩阵 $A = \left(\begin{array}{ccc}
  1 & 1 & 0\\
  1 & x & 0\\
  0 & 0 & 1
\end{array}\right)$ 其中两个特征值为 $\lambda_1 = 1$ 和 $\lambda_2
= 2$,则则 $x=$ \answer{B}
\options{2}
	{1}
	{0}
	{-1}
\end{question}



\begin{question}
二次型 $f = 4 x_1^2 - 2 x_1 x_2 + 6 x_2^2$ 对应的矩阵等于 \answer{C}
\options{$\left(\begin{array}{cc}
  4 & - 2\\
  - 2 & 6
\end{array}\right)$}
	{$\left(\begin{array}{cc}
  2 & - 2\\
  - 2 & 3
\end{array}\right)$}
	{$\left(\begin{array}{cc}
  4 & - 1\\
  - 1 & 6
\end{array}\right)$}
	{$\left(\begin{array}{cc}
  2 & - 1\\
  - 1 & 3
\end{array}\right)$}
\end{question}



\begin{question}
对任何一个本校男学生,以$A$表示他是大一学生,$B$表示他是大二学生,
则事件$A$和$B$是\answer{B}
\options[4]{对立事件}
	{互斥事件}
	{既是对立事件又是互斥事件}
	{不是对立事件也不是互斥事件}
\end{question}



\begin{question}
下列说法\uline{不正确}的是\answer{B}
\options[4]{大数定律说明了大量相互独立且同分布的随机变量的均值的稳定性}
	{大数定律说明大量相互独立且同分布的随机变量的均值近似于正态分布}
	{中心极限定理说明了大量相互独立且同分布的随机变量的和的稳定性}
	{中心极限定理说明大量相互独立且同分布的随机变量的和近似于正态分布}
\end{question}

% \vfill

\begin{question}
在数理统计中,对总体$X$和样本$(X_1,\cdots,X_n)$的说法哪个是\uline{不正确}的\answer{D}
\options{总体是随机变量}
	{样本是$n$元随机变量}
	{$X_1, \cdots, X_n$相互独立}
	{$X_1 = X_2 =\cdots = X_n$}
\end{question}

% \vfill

\begin{question}
样本平均数$\bar{X}$\uline{未必是}总体期望值$\mu$的\answer{A}
\options{最大似然估计}
	{有效估计}
	{一致估计}
	{无偏估计}
\end{question}


\begin{question}
已知集合 $A=\{(x,y)|x^2+y^2=1\}$ ,$B=\{(x,y)|y=x\}$,则 $A{\,\raisebox{0.8mm}{\scaleobj{0.55}{\bigcap}}\,}B$ 中元素的个数为
	\begin{tasks}(4)
		\task $3$ \task $2$ \task $1$ \task $0$
	\end{tasks}
\end{question}

\begin{question}
设复数 $z$ 满足 $(1+\mathrm{i}\,)z=2\,\mathrm{i}$,则 $|z|=$
	\begin{tasks}(4)
		\task $\frac{1}{2}$ \task $\frac{\sqrt{2}}{2}$ \task $\sqrt{2}$ \task $2$
	\end{tasks}
\end{question}



\begin{question}
$(x+y)(2x-y)^5$ 的展开式中 $x^3y^3$ 的系数为
	\begin{tasks}(4)
		\task $-80$ \task $-40$ \task $40$ \task $80$
	\end{tasks}
\end{question}

\begin{question}
已知双曲线 $C\colon\,\frac{x^2}{a^2}-\frac{y^2}{b^2}=1\,(a>0,b>0)$ 的一条渐近线方程为 $y=\frac{\sqrt{5}}{2}x$ ,且与椭圆
	$\frac{x^2}{12}+\frac{y^2}{3}=1$ 有公共焦点,则 $C$ 的方程为
	\begin{tasks}(4)
		\task $\frac{x^2}{8}-\frac{y^2}{10}=1$ \task $\frac{x^2}{4}-\frac{y^2}{5}=1$ \task $\frac{x^2}{5}-\frac{y^2}{4}=1$ \task $\frac{x^2}{4}-\frac{y^2}{3}=1$
	\end{tasks}
\end{question}

\begin{question}
设函数 $f(x)=\cos\Big(x+\frac{\pi}{3}\Big)$,则下列结论错误的是
	\begin{tasks}(2)
		\task $f(x)$ 的一个周期为 $-2\pi$ \task $y=f(x)$ 的图像关于直线 $x=\frac{8\pi}{3}$ 对称
		\task $f(x+\pi)$ 的一个零点为 $x=\frac{\pi}{6}$ \task  $f(x)$在 $\Big(\frac{\pi}{2},\pi\Big)$ 单调递减
	\end{tasks}
\end{question}

	
\begin{question}
已知圆柱的高为1,它的两个底面的圆周在直径为2的同一个球的球面上,则该圆柱的体积为
	\begin{tasks}(4)
		\task $\pi$ \task $\frac{3\pi}{4}$ \task $\frac{\pi}{2}$ \task 0
	\end{tasks}
\end{question}


\begin{question}
等差数列的首项为1,公差不为0. 若 $a_2$,$a_3$,$a_6$ 成等比数列,则前 6 项的和为
	\begin{tasks}(4)
		\task $-24$ \task $-3$ \task 3 \task 8
	\end{tasks}
\end{question}

\begin{question}
已知椭圆 $C\colon\,\frac{x^2}{a^2}+\frac{y^2}{b^2}=1\,(a>b>0)$,的左、右顶点分别为 $A_1$,$A_2$,且以线段 $A_1A_2$ 为直径的圆与直线 $bx-ay+2ab=0$ 相切,则 $C$ 的离心率为
	\begin{tasks}(4)
		\task $\frac{\sqrt{6}}{3}$ \task $\frac{\sqrt{3}}{3}$ \task $\frac{\sqrt{2}}{3}$ \task $\frac{1}{3}$
	\end{tasks}
\end{question}


\begin{question}已知函数 $f(x)=x^2-2x+a(e^{x-1}+e^{-x+1})$ 有唯一零点,则 $a=$
	\begin{tasks}(4)
		\task $-\frac{1}{2}$ \task $\frac{1}{3}$ \task $\frac{1}{2}$ \task 1
	\end{tasks}
\end{question}


\begin{question}
在矩形 $ABCD$ 中,$AB=1$,$AD=2$,动点 $P$ 在以点 $C$ 为圆心且与 $BD$ 相切的圆上.\\
	若 $\overrightarrow{{AP}}=\lambda\overrightarrow{{AB} }+\mu\overrightarrow{{AD}}$,则 $\lambda+\mu$ 的最大值为
	\begin{tasks}(4)
		\task 3 \task $2\sqrt{2}$ \task $\sqrt{5}$ \task 2
	\end{tasks}
\end{question}

\begin{question}
下列各排列哪个是偶排列 \answer{D}
\options{3712456}
	{36715284}
	{654321}
	{41253}
\end{question}



\begin{question}
若三阶行列式 $\left|\begin{array}{ccc}
  a_1 & a_2 & a_3\\
  2 b_1 - a_1 & 2 b_2 - a_2 & 2 b_3 - a_3\\
  c_1 & c_2 & c_3
\end{array}\right| = 2$,则 $\left|\begin{array}{ccc}
  a_1 & a_2 & a_3\\
  b_1 & b_2 & b_3\\
  c_1 & c_2 & c_3
\end{array}\right|=$ \answer{A}
\options{1}
	{-1}
	{2}
	{-2}
\end{question}



\begin{question}
已知矩阵 $A = \left(\begin{array}{ccc}
  1 & 1 & 0\\
  1 & x & 0\\
  0 & 0 & 1
\end{array}\right)$ 其中两个特征值为 $\lambda_1 = 1$ 和 $\lambda_2
= 2$,则则 $x=$ \answer{B}
\options{2}
	{1}
	{0}
	{-1}
\end{question}

% \vfill

\begin{question}
二次型 $f = 4 x_1^2 - 2 x_1 x_2 + 6 x_2^2$ 对应的矩阵等于 \answer{C}
\options{$\left(\begin{array}{cc}
  4 & - 2\\
  - 2 & 6
\end{array}\right)$}
	{$\left(\begin{array}{cc}
  2 & - 2\\
  - 2 & 3
\end{array}\right)$}
	{$\left(\begin{array}{cc}
  4 & - 1\\
  - 1 & 6
\end{array}\right)$}
	{$\left(\begin{array}{cc}
  2 & - 1\\
  - 1 & 3
\end{array}\right)$}
\end{question}



\begin{question}
对任何一个本校男学生,以$A$表示他是大一学生,$B$表示他是大二学生,则事件$A$和$B$是
\answer{B}
\options{对立事件}
	{互斥事件}
	{既是对立事件又是互斥事件}
	{不是对立事件也不是互斥事件}
\end{question}



\begin{question}
下列说法\uline{不正确}的是\answer{B}
\options{大数定律说明了大量相互独立且同分布的随机变量的均值的稳定性}
	{大数定律说明大量相互独立且同分布的随机变量的均值近似于正态分布}
	{中心极限定理说明了大量相互独立且同分布的随机变量的和的稳定性}
	{中心极限定理说明大量相互独立且同分布的随机变量的和近似于正态分布}
\end{question}



\begin{question}
在数理统计中,对总体$X$和样本$(X_1,\cdots,X_n)$的说法哪个是\uline{不正确}的\answer{D}
\options{总体是随机变量}
	{样本是$n$元随机变量}
	{$X_1, \cdots, X_n$相互独立}
	{$X_1 = X_2 =\cdots = X_n$}
\end{question}



\begin{question}
样本平均数$\bar{X}$\uline{未必是}总体期望值$\mu$的\answer{A}
\options{最大似然估计}
	{有效估计}
	{一致估计}
	{无偏估计}
\end{question} 