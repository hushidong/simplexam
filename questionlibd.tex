\begin{question}
我们班有26名男同学,23名女同学,按一小组最多六人来计算,全班最多可以分为几
    个小组?
\end{question}

\begin{solution}
  \begin{align*}
    26 + 23 = 49\text{(名)}
    49 \div 6 = 8 \cdots \cdots 1 \text{(个)}
  \end{align*}
  最多可以分为9个小组。
\end{solution}


\begin{question}
求不定积分$\displaystyle\int e^{2x}\,(\tan x+1)^2 dx$。
\end{question}

\begin{solution}
\everymath{\displaystyle}%
原式$=\int e^{2x}\,\sec^2 x dx+2\int e^{2x}\,\tan x dx$ \score{2}
\hspace{5em}${}=\int e^{2x}\,d(\tan x)+ 2\int e^{2x}\,\tan x dx$ \score{4}
\hspace{5em}${}= e^{2x}\,\tan x - 2\int e^{2x}\,\tan x dx+ 2\int e^{2x}\,\tan x dx$ \score{6}
\hspace{5em}${}= e^{2x}\,\tan x + C$ \score{8}
\end{solution}


\begin{question}
求过点$A(1,2,-1), B(2,3,0),C(3,3,2)$ 的三角形$\triangle ABC$ 的面积和它们确定的平面方程.
\end{question}

\begin{solution}
由题设$\overrightarrow{AB}=(1,1,1),\overrightarrow{AC}=(2,1,3)$, \score{2}
故$\overrightarrow{AB}\times \overrightarrow{AC}=\begin{vmatrix}
\vec{i}&\vec{j} &\vec{k}\\
1&1&1\\
2&1&3\\
\end{vmatrix}=(2,-1,-1)$, \score{4}
三角形$\triangle ABC$ 的面积为$S_{\triangle ABC}=\dfrac{1}{2}\big|\overrightarrow{AB}\times
\overrightarrow{AC}\big|=\dfrac{1}{2}\sqrt{6}.$ \score{6}
所求平面的方程为$2(x-2)-(y-3)-z=0$, 即$2x-y-z-1=0$ \score{8}			
\end{solution}


\begin{question}
计算四阶行列式 $A = \left|\begin{array}{cccc}
  0 & 1 & 2 & 3\\
  1 & 2 & 3 & 0\\
  2 & 3 & 0 & 1\\
  3 & 0 & 1 & 2
\end{array}\right|$ 的值.
\end{question}

\begin{solution}
$A = \left|\begin{array}{cccc}
    0 & 1 & 2 & 3\\
    1 & 2 & 3 & 0\\
    2 & 3 & 0 & 1\\
    3 & 0 & 1 & 2
  \end{array}\right| = \left|\begin{array}{cccc}
    0 & 1 & 2 & 3\\
    1 & 2 & 3 & 0\\
    0 & - 1 & - 6 & 1\\
    0 & - 6 & - 8 & 2
  \end{array}\right| = 1 \cdot (- 1)^{2 + 1} \left|\begin{array}{ccc}
    1 & 2 & 3\\
    - 1 & - 6 & 1\\
    - 6 & - 8 & 2
  \end{array}\right|$ \score{4}
\qquad $= -\left|\begin{array}{ccc}
    1 & 2 & 3\\
    0 & - 4 & 4\\
    0 & 4 & 20
  \end{array}\right| = - \left|\begin{array}{cc}
    - 4 & 4\\
    4 & 20
  \end{array}\right| = -(-4\cdot20-4\cdot4) = 96$ \score{8}
\end{solution}


\begin{question}
用配方法将二次型 $f = x_1^2 + 2 x_1 x_2 - 6 x_1 x_3 + 2 x_2^2 - 12
x_2 x_3 + 9 x^2_3$ 化为标准形 $f = d_1 y^2_1 + d_2 y^2_2 + d_3 y^2_3$ .
\end{question}

\begin{solution}
$f = x_1^2 + 2 x_1 x_2 - 6 x_1 x_3 + 2 x_2^2 - 12 x_2 x_3 + 9 x^2_3$ \par
\qquad$= x_1^2 + 2 x_1 (x_2 - 3 x_3) + (x_2 - 3 x_3)^2 + x_2^2 - 6 x_2 x_3 $ \par
\qquad$= (x_1 + x_2 - 3 x_3)^2 + x_2^2 - 6 x_2 x_3$ \score{3}
\qquad$= (x_1 + x_2 - 3 x_3)^2 + x_2^2 - 2 x_2 \cdot 3 x_3 + (3 x_3)^2 - 9x_3^2$ \par
\qquad$= (x_1 + x_2 - 3 x_3)^2 + (x_2 - 3 x_3)^2 - 9 x_3^2$ \score{6}
令$y_1 = x_1 + x_2 - 3 x_3, y_2 = x_2 - 3 x_3, y_3 = x_3$, \newline
则$f = y_1^2 + y_2^2 - 9y_3^2$为标准形.\score{8}
\end{solution}


\begin{question}
设每发炮弹命中飞机的概率是0.2且相互独立,现在发射100发炮弹.\par
 用切贝谢夫不等式估计命中数目$\xi$在10发到30发之间的概率.\par
 用中心极限定理估计命中数目$\xi$在10发到30发之间的概率.
\end{question}

\begin{solution}
$E\xi = n p = 100 \cdot 0.2 = 20, D\xi = n p q = 100 \cdot 0.2 \cdot 0.8 = 16$. \score{2}
 $P (10 < \xi < 30) = P (| \xi - E \xi | < 10) \ge 1 - \frac{D\xi}{10^2}
     = 1 - \frac{16}{100} = 0.84$. \score{4}
 $P (10 < \xi < 30) \approx \Phi_0 \left( \frac{30 - 20}{\sqrt{16}}\right)
     - \Phi_0 \left( \frac{10 - 20}{\sqrt{16}} \right)$ \score{6}
\qquad $= 2 \Phi_0 (2.5) - 1 = 2 \cdot 0.9938 - 1 =0.9876$ \score{8}
\end{solution}


\begin{question}
从正态总体$N(\mu,\sigma^2)$中抽出样本容量为16的样本,算得其平均数为3160,标准差为100.
试检验假设$H_0:\mu=3140$是否成立($\alpha = 0.01$).
\end{question}

\begin{solution}
 待检假设 $H_0 : \mu = 3140$. \score{1}
 选取统计量 $T = \frac{\bar{X}-\mu}{S / \sqrt{n}} \sim t(n-1)$. \score{3}
 查表得到 $t_{\alpha} = t_{\alpha} (n - 1) = t_{0.01} (15) =2.947$. \score{5}
 计算统计值 $t = \frac{\bar{x} - \mu_0}{s/\sqrt{n}} =\frac{3160-3140}{100/4} = 0.8$.\score{7}
 由于 $| t | < t_{\alpha}$, 故接受 $H_0$, 即假设成立. \score{8}
\end{solution}


\begin{question}
设数列$\{x_n\}$满足$x_1=\sqrt2$,$x_{n+1}=\sqrt{2+x_n}$.证明数列收敛,并求出极限.
\end{question}

\begin{solution}
事实上,由于$x_1<2$,且$x_k<2$时
\[x_{k+1}=\sqrt{2+x_k}<\sqrt{2+2}=2,\]
由数学归纳法知对所有$n$都有$x_n<2$,即数列有上界.
又由于
\[\frac{x_{n+1}}{x_n}=\sqrt{\frac{2}{x_n^2}+\frac{1}{x_n}}>\sqrt{\frac{2}{2^2}+\frac{1}{2}}=1,\]
所以数列单调增加.由极限存在准则II,数列必定收敛.\score{4}
设数列的极限为 $A$ ,对递推公式两边同时取极限得到
\[A=\sqrt{A+2}\]
解得$A=2$,即数列$\{x_n\}$的极限为$2$.\score{8}
\end{solution}



\begin{question}
设事件$A$和$B$相互独立,证明$A$和$\bar{B}$相互独立.
\end{question}

\begin{solution}
$P (A \cdot \bar{B}) = P (A - B) = P (A - A B)$ \score{2}
\qquad $= P (A) - P (A B) = P (A) - P (A) P (B)$ \score{4}
\qquad $= P (A) (1 - P (B)) = P (A) P (\bar{B})$ \score{6}
所以$A$和$\bar{B}$相互独立.\score{8}
\end{solution}


\begin{question}
计算四阶行列式 $A = \left|\begin{array}{cccc}
  0 & 1 & 2 & 3\\
  1 & 2 & 3 & 0\\
  2 & 3 & 0 & 1\\
  3 & 0 & 1 & 2
\end{array}\right|$ 的值。
\end{question}

\begin{solution}
$A = \left|\begin{array}{cccc}
    0 & 1 & 2 & 3\\
    1 & 2 & 3 & 0\\
    2 & 3 & 0 & 1\\
    3 & 0 & 1 & 2
  \end{array}\right| = \left|\begin{array}{cccc}
    0 & 1 & 2 & 3\\
    1 & 2 & 3 & 0\\
    0 & - 1 & - 6 & 1\\
    0 & - 6 & - 8 & 2
  \end{array}\right| = 1 \cdot (- 1)^{2 + 1} \left|\begin{array}{ccc}
    1 & 2 & 3\\
    - 1 & - 6 & 1\\
    - 6 & - 8 & 2
  \end{array}\right|$ \dotfill 4分\par
\qquad\qquad $= -\left|\begin{array}{ccc}
    1 & 2 & 3\\
    0 & - 4 & 4\\
    0 & 4 & 20
  \end{array}\right| = - \left|\begin{array}{cc}
    - 4 & 4\\
    4 & 20
  \end{array}\right| = -(-4\cdot20-4\cdot4) = 96$ \dotfill 8分
\end{solution}


\begin{question}
用配方法将二次型 $f = x_1^2 + 2 x_1 x_2 - 6 x_1 x_3 + 2 x_2^2 - 12
x_2 x_3 + 9 x^2_3$ 化为标准形 $f = d_1 y^2_1 + d_2 y^2_2 + d_3 y^2_3$ 。
\end{question}

\begin{solution}
$f = x_1^2 + 2 x_1 x_2 - 6 x_1 x_3 + 2 x_2^2 - 12 x_2 x_3 + 9 x^2_3$ \par
\qquad\qquad$= x_1^2 + 2 x_1 (x_2 - 3 x_3) + (x_2 - 3 x_3)^2 + x_2^2 - 6 x_2 x_3 $ \par
\qquad\qquad$= (x_1 + x_2 - 3 x_3)^2 + x_2^2 - 6 x_2 x_3$ \dotfill 3分 \par
\qquad\qquad$= (x_1 + x_2 - 3 x_3)^2 + x_2^2 - 2 x_2 \cdot 3 x_3 + (3 x_3)^2 - 9x_3^2$ \par
\qquad\qquad$= (x_1 + x_2 - 3 x_3)^2 + (x_2 - 3 x_3)^2 - 9 x_3^2$ \dotfill 6分\par
令$y_1 = x_1 + x_2 - 3 x_3, y_2 = x_2 - 3 x_3, y_3 = x_3$, \newline
则$f = y_1^2 + y_2^2 - 9y_3^2$为标准形。\dotfill 8分
\end{solution}


\begin{question}
设二元随机变量$(\xi, \eta)$的联合分布表为
\begin{tabular}{|l|l|l|l|}
  \hline
  $\xi \backslash \eta$ & -1 & 0 & 1\\
  \hline
  0 & 0 & 1/3 & 0\\
  \hline
  1 & 1/3 & 0 & 1/3\\
  \hline
\end{tabular}。\par
(1) 求关于$\xi$和$\eta$的边缘分布。\par
(2) 判断$\xi$和$\eta$的独立性。\par
(3) 判断$\xi$和$\eta$的相关性。
\end{question}

\begin{solution}
(1) 边缘分布为 \begin{tabular}{|l|l|l|}
  \hline
  $\xi$ & 0 & 1\\
  \hline
  $P$ & 1/3 & 2/3\\
  \hline
\end{tabular}, \ \begin{tabular}{|l|l|l|l|}
  \hline
  $\eta$ & -1 & 0 & 1\\
  \hline
  $P$ & 1/3 & 1/3 & 1/3\\
  \hline
\end{tabular}. \dotfill 2分 \par
(2) 由$P(\xi = 0, \eta = 0) = \frac{1}{3} \neq \frac{1}{9} = P(\xi = 0) P(\eta = 0)$,
知$\xi$和$\eta$不独立. \dotfill 4分 \par
(3) 由联合分布表求得$\xi \eta$的分布为 \begin{tabular}{|l|l|l|l|}
  \hline
  $\xi \eta$ & -1 & 0 & 1\\
  \hline
  $P$ & 1/3 & 1/3 & 1/3\\
  \hline
\end{tabular}.\dotfill 6分\par
因此有 $\cov(\xi, \eta) = E(\xi\eta) - E\xi E\eta = 0 -\frac{2}{3} \cdot 0 = 0$,
因此$\xi$和$\eta$不相关. \dotfill 8分
\end{solution}


\begin{question}
设随机变量$\xi \sim N (1, 4)$,求$P (- 1 < \xi < 5)$。
\end{question}

\begin{solution}
$P(-1<\xi<5) = \Phi_0\left(\frac{5-1}{2}\right) - \Phi_0\left(\frac{-1-1}{2}\right)$ \dotfill 2分 \par
\qquad $= \Phi_0 (2) - \Phi_0 (- 1)$ \dotfill 4分 \par
\qquad $= \Phi_0 (2) + \Phi_0 (1) - 1$ \dotfill 6分 \par
\qquad $= 0.9773 + 0.8413 - 1 = 0.8186$ \dotfill 8分
\end{solution}



\begin{question}
设每发炮弹命中飞机的概率是0.2且相互独立,现在发射100发炮弹。\par
(1) 用切贝谢夫不等式估计命中数目$\xi$在10发到30发之间的概率。\par
(2) 用中心极限定理估计命中数目$\xi$在10发到30发之间的概率。
\end{question}

\begin{solution}
$E\xi = n p = 100 \cdot 0.2 = 20, D\xi = n p q = 100 \cdot 0.2 \cdot 0.8 = 16$. \dotfill 2分 \par
(1) $P (10 < \xi < 30) = P (| \xi - E \xi | < 10) \geq 1 - \frac{D\xi}{10^2}
     = 1 - \frac{16}{100} = 0.84$. \dotfill 4分 \par
(2) $P (10 < \xi < 30) \approx \Phi_0 \left( \frac{30 - 20}{\sqrt{16}}\right)
     - \Phi_0 \left( \frac{10 - 20}{\sqrt{16}} \right)$ \dotfill 6分\par
\qquad $= 2 \Phi_0 (2.5) - 1 = 2 \cdot 0.9938 - 1 =0.9876$ \dotfill 8分
\end{solution}

\begin{question}
从正态总体$N(\mu,\sigma^2)$中抽出样本容量为16的样本,算得其平均数为3160,标准差为100。
试检验假设$H_0:\mu=3140$是否成立($\alpha = 0.01$)。
\end{question}

\begin{solution}
(1) 待检假设 $H_0 : \mu = 3140$. \dotfill 1分\par
(2) 选取统计量 $T = \frac{\bar{X}-\mu}{S / \sqrt{n}} \sim t(n-1)$. \dotfill 3分 \par
(3) 查表得到 $t_{\alpha} = t_{\alpha} (n - 1) = t_{0.01} (15) =2.947$. \dotfill 5分 \par
(4) 计算统计值 $t = \frac{\bar{x} - \mu_0}{s/\sqrt{n}} =\frac{3160-3140}{100/4} = 0.8$.\dotfill 7分 \par
(5) 由于 $| t | < t_{\alpha}$, 故接受 $H_0$, 即假设成立. \dotfill 8分
\end{solution}


\begin{question}
不使用矩阵可相似对角化的判别定理,直接用矩阵的运算和性质证明下面的矩阵$A
=\left(\begin{array}{cc}
  1 & 1\\
  0 & 1
\end{array}\right)$不能相似对角化,即不存在可逆矩阵$P$和对角阵$\Lambda$使得$P^{-1}AP=\Lambda$。
\end{question}

\begin{solution}
假设有$P = \left(\begin{array}{cc}
  a & b\\
  c & d
\end{array}\right)$使得$P^{-1}AP = \Lambda$,即$AP=P\Lambda$。\dotfill 2分\par
则有 $$\left(\begin{array}{cc}
  a + c & b + d\\
  c & d
\end{array}\right) = \left(\begin{array}{cc}
  1 & 1\\
  0 & 1
\end{array}\right) \left(\begin{array}{cc}
  a & b\\
  c & d
\end{array}\right) = \left(\begin{array}{cc}
  a & b\\
  c & d
\end{array}\right) \left(\begin{array}{cc}
  \lambda_1 & \\
  & \lambda_2
\end{array}\right) = \left(\begin{array}{cc}
  a \lambda_1 & b \lambda_2\\
  c \lambda_1 & d \lambda_2
\end{array}\right)$$ 因此有 $\left\{ \begin{array}{llll}
  a + c & = & a \lambda_1 & (1)\\
  b + d & = & b \lambda_2 & (2)\\
  c & = & c \lambda_1 & (3)\\
  d & = & d \lambda_2 & (4)
\end{array} \right.$ \dotfill 6分\par
由第1个和第3个方程消去$\lambda_1$,可以得到 $c^2 = 0$ 即 $c=0$;
由第2个和第4个方程消去$\lambda_2$,可以得到 $d^2 = 0$ 即 $d=0$。
因此矩阵$P$不可逆,矛盾。\dotfill 10分
\end{solution}


\begin{question}
设事件$A$和$B$相互独立,证明$A$和$\bar{B}$相互独立。
\end{question}

\begin{solution}
$P (A \cdot \bar{B}) = P (A - B) = P (A - A B)$ \dotfill 3分 \par
\qquad $= P (A) - P (A B) = P (A) - P (A) P (B)$ \dotfill 6分 \par
\qquad $= P (A) (1 - P (B)) = P (A) P (\bar{B})$ \dotfill 9分 \par
所以$A$和$\bar{B}$相互独立。\dotfill 10分
\end{solution}
